\subsection{TCP measurements}\label{sec_tcp_mes}

\subsection*{Purpose:}

To determine the maximum frequency that can be used with the TCP/IP protocol. To do so, we will measure the round time trip delay and jitter and look at those values as a function of the delay used in the sending loop.

\subsection*{Test equipment:}

The sbRIO 9636 is connected to a computer running Ubuntu 14.04 LTS via an Ethernet cable of 1 gpbs. The computer runs the ROS nodes that are used for the project, thus it communicates with both the Geomagic Touch and the sbRIO and it sends random setpoints to the sbRIO. The sbRIO uses the entire code of the project.

\subsection*{Procedure:}

\begin{enumerate}
	\item The sbRIO is started and connected to the computer.
	\item The ROS node is started.
	\item tshark is started.% using this command: tshark -i 1 -f tcp -a duration:20 -w ~/wireshark_capture
	\item Once the Wireshark recording is done the ROS node is stopped.
	\item The log files are retrieved to be analyzed using Matlab and Wireshark.
\end{enumerate}
Those steps are repeated for both the serial and parallel case for the following settings: 10 ms, 2 ms and 0 ms.


\subsection*{Measuring data:}

% \begin{center}
% 	$\begin{tabular}{c|ccccc}
% 		\text{Delay} & \text{Sending\newline frequency (Hz)} & \text{Receiving frequency (Hz)} & \text{Average delay (ms)} & \text{Jitter} & \text{Error rate (\%)}\\
% 		\hline
% 		10ms & 4 & 4 & 1 & 1 & 1 \\
% 		2ms & 4 & 4 & 1 & 1 & 1 \\
% 		0ms & 1 & 1 & 1 & 1 & 15 \\
% 	\end{tabular}$
% 	\captionof{table}{TCP}
% \end{center}
\todo{Find a way to represent what is desbribed in Results, as of now we have the wireshark files}
\subsection*{Results:}

When the frequency is increased above 100Hz, instead of sending more packets it sends bigger packets in which two \gls{JSON} files are grouped. The more the frequency is increased, the bigger the packets.

\subsection*{Conclusion:}

The communication does not have the desired behavior when the frequency is increased. This may be due to the default behavior of TCP or to the default behavior of the JSON Stream. It was decided not to look further into this subject as this problem does not appear in the new communication protocol.