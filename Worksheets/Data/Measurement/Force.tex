\section{Force measurements for the end-effector} % \label{app:...}

As the Endowrist tool is highly nonlinear, measurement of how the end effector responds to external force should be made. These measurements are made for the parameters estimation of the EndoWrist model.  

\subsection*{Purpose:}
The purpose is to measure the relationship between the current consumption and force acting at the end-effector. Furthermore these measurements are used in the parameter estimation for the model of the EndoWrist.

The force estimations are done on three different setups:
\begin{itemize}
\item Pull/push of one clamp on the end-effector (yaw),
\item Pull/push of two clamps where the force is applied in the same direction (yaw),
\item Pull/push of the pitch on the end-effector (pitch)
\item and the rotation of the end-effector (roll).
\end{itemize}

In the following subsections the different tests defined in the itemizer above are made. For making the force measurements, item one to three, different attachment to a load-cell is made. These can be seen on \figref{fig:Overview_endowrist_attachment}.

\begin{figure}[H]
	\centering
	\begin{subfigure}{.32\textwidth}
		\centering
		\vspace{-12pt}
		\includegraphics[width=\linewidth]{One_clamp.jpg}
		\caption{One clamp is attached to the load cell, so the pull/push force of one clamp can be made.}
		\label{fig:one_clamp}
	\end{subfigure}
	\begin{subfigure}{.32\textwidth}
		\centering
		\includegraphics[width=\linewidth]{Two_clamp.jpg}
		\caption{The clamps are put together and connected to the load-cell such that the pull/push force can be measured.}
		\label{fig:two_clamp}
	\end{subfigure}
	\begin{subfigure}{.32\textwidth}
		\centering
		\includegraphics[width=\linewidth]{Pitch_clamp.jpg}
		\caption{The end-effector are turned $90^\circ$ and attached to the load-cell, such that the pitch force can be measured.}
		\label{fig:pitch_force}
	\end{subfigure}
\caption{Attachments between the end-effector and the load-cell used for force measurements}
\label{fig:Overview_endowrist_attachment}
\end{figure}


\input{Data/Measurement/Pull_clamp}

\input{Data/Measurement/Push_clamp}

\input{Data/Measurement/Pull_Two_clamp}

\input{Data/Measurement/Push_Two_clamp}

\subsection{Pitch inwards and outwards force} % \label{app:...}
This test uses the test setup seen on \figref{fig:pitch_force}. The positive force is defined as a downwards direction on the load-cell.


\subsection*{Test equipment:}
\begin{itemize}
\item Endowrist model 420093 (AAU number: \#4).
\item Maxon 110160 motor with attached Maxon gearhead 110356 and Maxon encoder 201937.
\item Load cell rate for 1 kg of force \cite{Load_cell_1kg}.
\item HX711 - Load cell amplifier \cite{HX711}
\end{itemize}

\subsection*{Procedure:}
The following procedure was made:\\
Downwards force measurements:
\begin{enumerate}
\item The end-effector is rotated $90^\circ$ and attached perpendicular to the load cell. 
\item The scale is reset to zero.
\item Current is applied to the motor which control the pitch of the end-effector, at different current levels and the force is measured (downwards direction).
\item Current is increased until 1200 mA is applied.
\end{enumerate}
Step two to four is repeated five times, where the current and force is measured in respect to each other. 

Upwards force measurements:
\begin{enumerate}
\item The end-effector is rotated $90^\circ$ and attached perpendicular to the load cell. 
\item The scale is reset to zero.
\item Current is applied to the motor which control the pitch of the end-effector, at different current levels and the force is measured (upwards direction).
\item Current is increased until 1200 mA is applied.
\end{enumerate}
Step two to four is repeated five times, where the current and force is measured in respect to each other. 

\subsection*{Measuring data:}
Six of the data measurements can be seen on \figref{fig:pitch_down} and \figref{fig:pitch_up}.

\begin{figure}[H]
\centering
\input{Data/Measurement/EndoWrist_Measurements/Force/pitch_down}
\caption{Force measurements for the pitch in an downwards direction.}
\label{fig:pitch_down}
\end{figure}

\begin{figure}[H]
\centering
\input{Data/Measurement/EndoWrist_Measurements/Force/pitch_up}
\caption{Force measurements for the pitch in an upwards direction.}
\label{fig:pitch_up}
\end{figure}

%\figref{endo_force_mes}. 
%\eqref{eq:linear_force_endo}.

% \begin{equation}
% \text{y} = 0.0028 \cdot \text{x} -0.8259 
% \label{eq:linear_force_endo}
% \end{equation} 



\subsection*{Results:}
From \figref{fig:pitch_down} and \figref{fig:pitch_up} a similar pattern between the measurements can be seen. From each measurement it can be seen that the force growth for the different measurements are similar. Force is generated on the end-effector from the same current start point on both upwards and downwards measurements. It can be seen that the current is decreasing for each measurements. This is due to the controller implemented on the motor controller as it gets a setpoint and current is applied until the setpoint is reached and then decreased.
%It can be seen from the graph on \figref{endo_force_mes} that the force on the end-effector is highly nonlinear. The friction from the gearing and the Endowrist does that the force first has an exponential growth at the start. Around the 800 mA and 1200 mA step it can be seen that a drop in force is happening. What causes this drop is not identified but it can be seen that it appears for all the data sequences. \todor{better explanation?}


%\input{Data/Measurement/Endo_force_measure.tex}
\subsection*{Uncertainties of measurement:}
\begin{itemize}
\item Not 100 \% orthogonal force to the load cell.
\item Input/Output impedance of sensors have a $\pm 10 \%$ tolerance.
\item Movement of test setup when force is generated.
\end{itemize}

\subsection*{Conclusion:}
Force is generated at the end-effector when a current higher than 100 mA is applied to the motor. This is identical for all measurements. 

\input{Data/Measurement/Pitch_push}

\subsection{EndoWrist roll measurement}% \label{app:...}
The test setup for this measurement can be seen on \todo{figref and a picture below}

\subsection*{Test equipment:}
\begin{itemize}
\item Endowrist model 420093 (AAU number: \#4).
\item Maxon 110160 motor with attached Maxon gearhead 110356 and Maxon encoder 201937.
\item Torque sensor - Holger clasen: MWA-W8-1-P (AAU number: LBNR 08931).
\item Active low pass filter set to 100 Hz and gain 1 (AAU number: C2-104-H1).
\item Agilent 54621A oscilloscope (AAU number: 56684)
\end{itemize}

\subsection*{Procedure:}
The following procedure was made for the push force measurements:
\begin{enumerate}
\item The carbon stick of the EndoWrist is attached to the torque sensor. 
\item The torque sensor is connected to the low pass filter and the low pass filter to the oscilloscope.
\item Current is applied or increased to the motor which control the roll of the end-effector, with different current steps and the output from the torque sensor is noted.
\item Current is increased until 1200 mA is applied.
\end{enumerate}
Step three and four is repeated four times, where the current and torque is noted in respect to each other.. 
\todo{update n times}

\subsection*{Measuring data:}
The data from the measurements can be seen on \todo{Picture}

\input{Data/Measurement/roll_force.tex}

%\figref{endo_force_mes}. 
%\eqref{eq:linear_force_endo}.

% \begin{equation}
% \text{y} = 0.0028 \cdot \text{x} -0.8259 
% \label{eq:linear_force_endo}
% \end{equation} 



\subsection*{Results:}
\todo{result}


%It can be seen from the graph on \figref{endo_force_mes} that the force on the end-effector is highly nonlinear. The friction from the gearing and the Endowrist does that the force first has an exponential growth at the start. Around the 800 mA and 1200 mA step it can be seen that a drop in force is happening. What causes this drop is not identified but it can be seen that it appears for all the data sequences. \todor{better explanation?}


%\input{Data/Measurement/Endo_force_measure.tex}

\subsection*{Uncertainties of measurement:}
\begin{itemize}
\item Gain of low pass filter could deviate from 1.
\end{itemize}

\subsection*{Conclusion:}
It can be seen on figure \figref{fig:torque_measurement} that the roll-torque of the EndoWrist has a linear growth from approx. 100 mA and up. The parameters for this estimation can therefore be expressed as y = A$\cdot$x+B.






% \begin{figure}[H]
% 	\centering
% 	\begin{subfigure}{.45\textwidth}
% 		\centering
% 		\vspace{24pt}
% 		\includegraphics[width=\linewidth]{overall_force.jpg}
% 		\caption{The entire test setup. From the right, a load cell for measuring the downwards force, a piece of wood for stiffening the Endowrist and keeping it it place and the Endowrist holder with motors.}
% 		\label{fig:entire_force_testsetup}
% 	\end{subfigure}
% 	\begin{subfigure}{.45\textwidth}
% 		\centering
% 		\includegraphics[width=\linewidth]{endeffector_force.jpg}
% 		\caption{The load cell with the attached end-effector.}
% 		\label{fig:endeffector_force}
% 	\end{subfigure}
% \caption{Test setup for the force estimation of the end-effector.}
% \label{fig:Overview_force}
% \end{figure}



