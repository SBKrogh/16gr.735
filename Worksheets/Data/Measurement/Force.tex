\section{Measurement of external force at end effector} % \label{app:...}

As the Endowrist tool is highly nonlinear, measurement of how the end effector responds to external force should be made.

\subsection*{Purpose:}
To measure the relationship between the current and the force at the end-effector.

This test was made from engineering a simple setup where force could be measured. It is however a rough estimate as the construction is not professionally made. 

The test setup can be seen on \figref{fig:Overview_force} 

\begin{figure}[H]
	\centering
	\begin{subfigure}{.45\textwidth}
		\centering
		\vspace{24pt}
		\includegraphics[width=\linewidth]{overall_force.jpg}
		\caption{The entire test setup. From the right, a load cell for measuring the downwards force, a piece of wood for stiffening the Endowrist and keeping it it place and the Endowrist holder with motors.}
		\label{fig:entire_force_testsetup}
	\end{subfigure}
	\begin{subfigure}{.45\textwidth}
		\centering
		\includegraphics[width=\linewidth]{endeffector_force.jpg}
		\caption{The load cell with the attached end-effector.}
		\label{fig:endeffector_force}
	\end{subfigure}
\caption{Test setup for the force estimation of the end-effector.}
\label{fig:Overview_force}
\end{figure}

\subsection*{Test equipment:}
\begin{itemize}
\item Endowrist model 420093 (AAU number: \#4).
\item Maxon 110160 motor with attached Maxon gearhead 110356 and Maxon encoder 201937.
\item Block of wood for stabilizing the EndoWrist.
\item Load cell rate for 1 kg of force \cite{Load_cell_1kg}.
\item HX711 - Load cell amplifier \cite{HX711}
\end{itemize}

\subsection*{Procedure:}
The following procedure was made for the force estimation:
\begin{enumerate}
\item The end-effector is attached perpendicular to the load cell. 
\item The scale is reset to zero.
\item Current is applied or increased to the motor which control the yaw of the end-effector, with steps of 120 mA and the force is measured.
\item Current is increased until 1200 mA is applied.
\end{enumerate}
Step two to four is repeated seven times, where the data is logged. 


\subsection*{Measuring data:}
The data from the measurements can be seen on \figref{endo_force_mes}. 
%\eqref{eq:linear_force_endo}.

% \begin{equation}
% \text{y} = 0.0028 \cdot \text{x} -0.8259 
% \label{eq:linear_force_endo}
% \end{equation} 



\subsection*{Results:}
It can be seen from the graph on \figref{endo_force_mes} that the force on the end-effector is highly nonlinear. The friction from the gearing and the Endowrist does that the force first has an exponential growth at the start. Around the 800 mA and 1200 mA step it can be seen that a drop in force is happening. What causes this drop is not identified but it can be seen that it appears for all the data sequences. \todor{better explanation?}


\input{Data/Measurement/Endo_force_measure.tex}

\subsection*{Uncertainties of measurement:}
\begin{itemize}
\item Weight calibration.
\item Not 100 \% orthogonal force to the load cell.
\item Input/Output impedance of sensors have a $\pm 10 \%$ tolerance.
\end{itemize}

\subsection*{Conclusion:}
It can be seen that the force on the end-effector has an exponential growth.

This test setup made it possible to make a rough measurements of the force at the end-effector.

