\input{preamble}	
\raggedbottom
\begin{document}

%%% Tikz Magic %%%
\include{tikz_magi}

% Forindhold - Kun romer tal

\frontmatter

%Formalia
%%main for introduktion

%\includepdf[pages={1}]{rapport/introduktion/forsiden.pdf}

%\includepdf[pages={1}]{forside/forsiden.pdf}

\thispagestyle{empty}

\begin{center}

\vspace*{\fill}

\textsc{\LARGE Aalborg University}\\[1.0cm]

\HRule \\[0.4cm]
{ \HUGE \bfseries  Automatic container model crane \\[0.5cm] }

\HRule \\[1.5cm]%

\begin{figure}[H]
\centering
\includegraphics[width=1\textwidth]{rapport/billeder/Crane}
\end{figure}

\begin{minipage}{0.4\textwidth}
\begin{flushleft} \large
Electronic og IT:\\
Third year of study
\end{flushleft}
\end{minipage}
\begin{minipage}{0.4\textwidth}
\begin{flushright} \large
Gruppe: \\
EIT-633
\end{flushright}
\end{minipage}

\vspace*{\fill}

\textsc{\Large Student report}\\[1.0cm]

{\large 25. May 2016}

\end{center}

%\phantomsection
\pdfbookmark[0]{Titelblad}{titelblad}
\thispagestyle{empty}

\begin{minipage}[t]{0.48\textwidth}
\vspace*{-25pt}			%\vspace*{-9pt}

\begin{figure}[H] 
\includegraphics[width=0.95\textwidth]{rapport/introduktion/aau_logo1}
\end{figure} 
\end{minipage}
\hfill
\begin{minipage}[t]{0.48\textwidth}
{\small 
\textbf{Third year of study}  \\
Electronic og IT \\
Fredrik Bajers Vej 7 \\
DK-9220 Aalborg East, Denmark\\
http://www.es.aau.dk}
\end{minipage}


\vspace*{1cm}

\begin{minipage}[t]{0.48\textwidth}
\textbf{Topic:} \\[5pt]\bigskip\hspace{2ex}BSc Project Control Engineering 

\textbf{Project:} \\[5pt]\bigskip\hspace{2ex}
P6-project

\textbf{Project time:} \\[5pt]\bigskip\hspace{2ex}
February 2016 - May 2016

\textbf{Projectgroup:} \\[5pt]\bigskip\hspace{2ex}
16gr633	

\textbf{Participants:} \\[5pt]\hspace*{2ex}
Daniel Bähner Andersen \\\hspace*{2ex}
Nicolaj Vinkel Christensen \\\hspace*{2ex}
Ralf Victor Lømand R. Christiansen\\\hspace*{2ex}
Simon Bjerre Krogh \\\hspace*{2ex}
Thomas Holm Pilgaard \\

\textbf{Supervisor:} \\[5pt]\hspace*{2ex}
Tom S. Pedersen \\\hspace*{2ex}
Kirsten M. Nielsen \\\bigskip\hspace{2ex}

\vspace*{3.5cm}

\textbf{Circulation: 8} \\
\textbf{Number of pages: 152}\\
\textbf{Appendix: 73 + CD} \\
\textbf{Completed 25-05-2016}\\
\end{minipage}
\hfill
\begin{minipage}[t]{0.483\textwidth}
\textbf{Synopsis:} \\[5pt]
\fbox{\parbox{7cm}{\bigskipThis project covers the development of a controller system for a model container crane. The goal of this project is to automatically relocate containers fast, reliably and efficiently. To achieve this, the hardware platform has been improved and is now based on digital motor drivers together with a FPGA as the main controlling unit. The control system is based on multiple controllers derived by the use of classical control methods, namely the root locus method. The system models describing the behavior of the crane are derived using a combination of Newtons law's and free body diagrams to describe this graphically. An issue discovered during the project, is that the static friction is not constant along the axes of the setup. In order to compensate for the static friction a current offset is implemented. However, this yields issues as the static friction is not constant. The controller system for relocating containers has been proved functional.Additionally an interface has been implemented, this allows to relocate containers from eight different positions along the x-axis and three locations along the y-axis.          \bigskip}}
\end{minipage}

\vfill

%{\footnotesize\itshape Rapportens indhold er frit tilgængeligt, men offentliggørelse (med kildeangivelse) må kun ske efter aftale med forfatterne. Forside billed }

\chapter*{Preface}

This project comprises of implementing a functional controller system for the provided miniature model crane. This is conducted with the purpose of relocating containers.  
A basic understanding of electronics is expected of the reader. 
The report is constructed with different chapters, each covering a main part of the entire project. 
First, a thorough analysis of the crane setup and the modernizations carried out to the hardware platform is presented. 
Furthermore, the construction of models describing the crane and verification of these is presented. 
The second part consists of designing controllers based on requirements derived from an
earlier chapter. The controllers are implemented and tested. Finally an acceptance test is performed and this is documented in the appendix. 
All of the raw test information, test programs and relevant datasheets are stored on the enclosed CD.
Regarding figures without a source reference, these are created specifically for this report.  

A big thanks to the group's supervisors Tom S. Pedersen and Kirsten M. Nielsen for help with discussing possible solutions and giving insight into the current solutions in use, concerning classical control theory. 

Thanks to Rasmus Viking Lømand Ravgård Christiansen, bachelor in mechanical engineering, for helping creating a \textbf{Solid Work} \cite{Solid_Work} model for parts of the crane. 

\begin{flushright}
Aalborg University, 25th of May 2016
\end{flushright}





\vfill

\begin{table}[H]
	\centering
		\begin{tabular}{c c }
			\underline{\phantom{mmmmmmmmmmmmmmmmmmm}}       & \underline{\phantom{mmmmmmmmmmmmmmmmmmm}} \\
			Daniel Bähner Andersen			 & Nicolaj Vinkel Christensen  \\
			\textit{dban13@student.aau.dk} & \textit{nvch13@student.aau.dk}\\
			&\\
			&\\
			\underline{\phantom{mmmmmmmmmmmmmmmmmmm}}       & \underline{\phantom{mmmmmmmmmmmmmmmmmmm}} \\
			Ralf Victor Lømand R. Christiansen			 & Simon Bjerre Krogh\\
			\textit{rvlr13@student.aau.dk} & \textit{skrogh13@student.aau.dk} \\
			&\\
			&\\	
		\end{tabular}
		\begin{tabular}{c c c}
			& \underline{\phantom{mmmmmmmmmmmmmmmmmmm}} 	& \\
			& Thomas Holm Pilgaard 					& \\
			& \textit{tpilga12@student.aau.dk}		& \\
		\end{tabular}
\end{table}




 %%%%%%%     <->    Front page, synopsis and preface included in this
\include{rapport/Start_Templates/The_Four_W}
\include{rapport/Start_Templates/Test_template}

\tableofcontents*

% Hovedindhold - nummereres fra ''side 1 af sidste side''b
\mainmatter 
\makeevenfoot{AAU}{\thepage \text{} of \pageref{LastPage}}{}{}
\makeoddfoot{AAU}{}{}{\thepage \text{} of \pageref{LastPage}}
%%%%%%%%%%%%%%%%%%%%%%%%%%%%%%%%%%%%%%%%%%%%%%%%%%%%%%%%%%%%%%%
% Delete this before turning in


%%%%%%%%%%%%%%%%%%%%%%%%%%%%%%%%%%%%%%%%%%%%%%%%%%%%%%%%%%%%%%%
%  Include sections here   %

\chapter{Kinematic for the Geomagic touch} % \label{app:...}

bla bla bla

\begin{table}[h!]
\centering
\label{my-label}
\begin{tabular}{|l|l|l|l|l|}
\hline
 &  &  &  &  \\ \hline
 &  &  &  &  \\ \hline
 &  &  &  &  \\ \hline
 &  &  &  &  \\ \hline
 &  &  &  &  \\ \hline
 &  &  &  &  \\ \hline
 &  &  &  &  \\ \hline
 &  &  &  &  \\ \hline
 &  &  &  &  \\ \hline
\end{tabular}
\caption{Pleas give me some kind of information!}
\end{table}



All surgical interventions with the Da Vinci robot require some sort of feedback for the user, in our work we are covering force/haptic feedback.
This type of feedback translates the forces acting on the surgical instrument to the haptic controller in some manner.
The haptic controller used in this work is the Phatnom Omni (aka Geomagic Touch) developed by Sensable Technologies.
It is one of the most cost effective haptic controllers currently on the market.

In order for useful feedback to be created, we need to be able to control the direction of the force created by the Phantom Omni.
A logical first step would be to derive the relationship between joint coordinates and end-effector position (the end-effector in this case being the point of contact between the user and device). 
Once we have this, it is possible to develop other constructs such as Jacobian and dynamic equations.

We derive the forward kinematics of the Phantom omni using the DH-algorithm for simplicity. Also, since the actuated joints only control the position, and the orientation joints are concentrated in a small space, we can separate the FK matrices into translational and rotational parts.

\begin{align}
\begin{tabular}{ |p{2cm}|p{2cm}|p{2cm}|p{2cm}|  }
 \hline
 \multicolumn{4}{|c|}{Phantom Omni DH parameters} \\
 \hline
 a &d &$\theta$ &$\alpha$ \\
 \hline
 0          & 0   &$q_{1}$&   0\\
 0          &   0           & $q_{2}$ &$\frac{\pi}{2}$\\
 A         & 0   & $q_{3}$    &  0\\
 0          & D             & $q_{4}$     & $\frac{\pi}{2}$\\
 0& 0 & $q_{5}$ &-$\frac{\pi}{2}$\\
  0& 0 & $q_{6}$ &$\frac{\pi}{2}$\\
 \hline
\end{tabular}
\end{align}



\begin{equation}
\mathbf{T_0^1} =\begin{bmatrix} \cos\theta_{1}& 0 & \sin\theta_{1} & 0\\
\sin\theta_{1}& 0 & -\cos\theta_{1} & 0\\
0 & 1 & 0 & d_{1}\\
0 & 0 & 0 & 1\\
\end{bmatrix}
\end{equation}







%%%%%%%%%%%%%%%%%%%%%%%%%%%%%%%%%%%%%%%%%%%%%%%%%%%%%%%%%%%%%%%
%  Bilag  %
\appendix

\clearpage

\bibliography{bibliography}

\end{document}