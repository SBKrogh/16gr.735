\subsection{Protocol}\label{sec:Protocol}

Finding a communication protocol faster than TCP/IP is possible as the safeties in the protocol tend to slow down the overall speed of the communication\todo{we need a ref for this}. Furthermore, some of the safeties like congestion control are irrelevant in our case since we are directly connected to the destination. Flow control is not necessary as the need for flow control implies that a delay is created on the receiving side which would induce delay in the entire system. Reliable transport and ordered delivery are a problem, since retransmitting data is useless in our case as the data can be deprecated. Instead, it is much more interesting to receive new data. 

The protocol chosen for this project is \gls{UDP}. This is done because its a faster protocol than TCPI/IP and the project goal is to reduce the latency in communication between the computer running ROS and the sbRIO board. It is however less reliable than TCP/IP but it is deemed irrelevant to some extent as a new package with data should be send as soon as possible. This also solve the problem with storing old data in the transmitter buffer which could be obsolete. Even though it is deem irrelevant that \gls{UDP} is a less reliable protocol and package lose can appear, it is still necessary to include safety in the system if an connection error should appear.  


% As stated in \secref{sec:com_ROS_sbRIO}, the current communication runs at a frequency of a 100 Hz and to do a force feedback, a minimum of 300 Hz is needed, see the introduction to \chapref{cha:improvement}. This means that it is required to find a way to speed up the communication in order to do force feedback. Speeding up the communication could be done in different ways, where three is stated below

% \begin{itemize}
% 	\item Find a faster communication protocol,
% 	\item Minimize the amount of data transmitted,
% 	\item Implement faster hardware.	
% \end{itemize}

% As stated in \secref{sec:com_ROS_sbRIO}, the current communication is set to run at a frequency of 100 Hz and to do a force feedback we need more than 300 Hz, see \secref{sec:min_speed}. This means that it is required to find a way to speed up the communication in order to do a force feedback. Speeding up the communication could be done by three different ways:

% \begin{itemize}
% 	\item Find a faster communication protocol,
% 	\item Minimize the amount of data transmitted,
% 	\item Implement faster hardware.	
% \end{itemize}


% Implementing faster hardware does not match with the goals of our project one of them being to use only the existing hardware.\todo{This is not written anywhere?}


