%\section{Minimum refresh speed}\label{sec:min_speed}

The refresh rate for the force feedback has an influence on the transparency of the controller. An ideal situation is that there is no delay in the system and the system is perfectly modelled, thus give the user a direct feedback of the forces the tool is exposed to.\\ 
The force can also be scaled if high precision tasks need to be performed, thus increasing the resistance the operator is fed through the feedback.

\todo{maybe scaled? and Update rate vs. delay. Also jitter should be considered} 


As one of the goals for this project is to implement a force feedback controller, the requirement for the force feedback speed should be met. Thus the bandwidth of the communication speed should be increased.

% From the article \textit{The role of haptics in medical training simulators: a survey of the state of the art}\cite{coles2011role}, it is said that the minimum refresh rate of haptic feedback is wildly debated to be between 300 Hz and 600 Hz, but for a realistic force feedback it is commonly accepted to be at least 1000 Hz, thus it is not defined. \todo{This is just marked with a "!" sign by Christoffer}

From \secref{sec:prev_communication} the communication between the sbRIO and the computer was in the previous setup running at 100 Hz, which does not meet the requirement of a 1000 Hz. Therefore the speed should be raised with at least a factor of ten. It is however preferable to exceed this so the system does not run on the limit, thus having the ability to make complex calculations.


\subsection*{Approach}
Speeding up the communication could be done in different ways, where three are stated below

\begin{itemize}
	\item Find a faster communication protocol,
	\item Minimize the amount of data transmitted,
	\item Implement faster hardware.	
\end{itemize}

Implementing faster hardware is not a possibility in this case as one of the goals for this project is to only use the existing hardware.
However implementing a new communication protocol and minimizing the package size is possible.\\
\todo{What will we test in the following sections..small tail to this introduction would be nice for the flow }
% Implementing faster hardware does however not match with the goals of this project as one of them being to use only the existing hardware.\todo{This is not written anywhere?} 