\section{Summary}
As mentioned before, a fully featured DaVinci robot has four arms with 6-7 actuated \gls{DOF} in total when the EndoWrist is included.
%Since the robot has 4 arms, there are 4 instruments.
%Although our setup controls only 4 motors, in funcionality it is equivalent to one DaVinci arm.
Although the test setup only controls four motors, it can manipulate the surgical tool itself in the same way the da Vinci robot does. The missing features are the ones related to the hand of the robot holding the tool, see \secref{sec:da_vin_rob}.

The sbRIO board controls the test setup and as such represents the onboard computer on the DaVinci robot.
In order to perform higher level functions such as force feedback control, it is necessary to remotely handle data and send high-level commands.
This is handled by an external computer system that is connected to the Geomagic Touch device.

The sbRIO board and the Geomagic Touch both communicate with the computer using UDP. The computer handles the force estimation using a dynamical model of the test setup (or EndoWrist, more precisely), this is vital for force feedback.
Connecting software components responsible for communicating with hardware and the ones responsible for the control algorithm and estimation is necessary.
For this purpose we use the Robot Operating System (ROS), which uses a network architecture to share data between components via data streams. Go to \secref{sec:def_ROS} for a short introduction to ROS.

A general overview of the system that will be developed can be seen on \figref{sec:summering_model}.


\begin{figure}
\begin{tikzpicture}
\draw (-1.5,-1.5) rectangle (13.5,1.5);


\node[fill = {rgb:red,1;green,2;blue,5},box] (Opt) at (0,0) {Operator};
\node[fill = {rgb:red,1;green,2;blue,5},box] (Geo) at ($(3,0) + (Opt)$) {Geomagic\\touch};
\node[fill = {rgb:red,1;green,2;blue,5},box] (ros) at ($(3,0) + (Geo)$) {Robotic\\operating\\system};
\node[fill = {rgb:red,1;green,2;blue,5},box] (davin) at ($(3,0) + (ros)$) {Da Vinci\\robot};
\node[fill = {rgb:red,1;green,2;blue,5},box] (end) at ($(3,0) + (davin)$) {Endowrist};


\draw[->, ultra thick] ([yshift=0.3cm]Opt.east) -- ([yshift=0.3cm]Geo.west);
\draw[->, ultra thick] ([yshift=0.3cm]Geo.east) -- ([yshift=0.3cm]ros.west);
\draw[->, ultra thick] ([yshift=0.3cm]ros.east) -- ([yshift=0.3cm]davin.west);
\draw[->, ultra thick] ([yshift=0.3cm]davin.east) -- ([yshift=0.3cm]end.west);


\draw[<-, ultra thick] ([yshift=-0.3cm]Opt.east) -- ([yshift=-0.3cm]Geo.west);
\draw[<-, ultra thick] ([yshift=-0.3cm]Geo.east) -- ([yshift=-0.3cm]ros.west);
\draw[<-, ultra thick] ([yshift=-0.3cm]ros.east) -- ([yshift=-0.3cm]davin.west);
\draw[<-, ultra thick] ([yshift=-0.3cm]davin.east) -- ([yshift=-0.3cm]end.west);

\node at (1.5,1) {Position};
% \node at (4.5,1) {Position};
% \node at (7.5,1) {yes};
% \node at (10.5,1) {yes};

\node at (1.5,-1) {Force};
% \node at (4.5,1) {Position};
% \node at (7.5,1) {yes};
% \node at (10.5,1) {yes};
\end{tikzpicture}
\caption{Overall system with feedback in both direction}
\end{figure}




\todo{The figure: We need to make sure that we actually uses the data required from the ES.}

\todo{Don't know if the text below the figure should be included or not. Thought it was needed as the figure else is standing without explanation}

From left to the right on \figref{sec:summering_model}.\\ 
The operator are setting the Geomagic Touch position, which then are send to the main computer. The computer sends the position to the embedded system, which handles the position of the EndoWrist.

From right to left on \figref{sec:summering_model}.\\
The embedded system sends back the position, velocity and current applied to the motor. The computer computes a force from the received data and sends it to the Geomagic Touch, which generate a force in a counter direction of the force applied by the human.
