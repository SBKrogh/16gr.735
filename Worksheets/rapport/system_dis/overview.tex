\section{Overview}
As mentioned before, a fully featured DaVinci robot has four arms with 6-7 \gls{DOF} in total, when the Endowrist instrument included.
%Since the robot has 4 arms, there are 4 instruments.
%Although our setup controls only 4 motors, in funcionality it is equivalent to one DaVinci arm.
Although our setup controls only 4 motors, it can manipulate the surgical tool itself in the same way the da Vinci robot does. The missing features are the one related to the hand of the robot holding the tool.

The sbRio board controls the test setup and as such represents the onboard computer on the DaVinci robot.
In order to perform higher level functions such as force feedback control, it is necessary to remotely handle data and send high-level commands.
This is handled by an external computer system that is connected to the Geo magic touch device.

The sbRIO board and the Geomagic Touch both communicate with the computer using UDP.
The computer also performs force estimation using a dynamical model of the test setup (or EndoWrist, more precisely), this is vital for force feedback.
In order to connect software components responsible for communicating with hardware and the ones responsible for the control algorithm and estimation.
For this purpose we use the Robot Operating System (ROS), which uses a network architecture to share data between components via data streams.

\begin{figure}[H]
\begin{tikzpicture}
\draw (-1.5,-2.5) rectangle (13.5,2.5);


\node[box] (Opt) at (0,0) {Operator};
\node[box] (Geo) at ($(3,0) + (Opt)$) {Geomagic\\touch};
\node[box] (ros) at ($(3,0) + (Geo)$) {Robotic\\operating\\system};
\node[box] (davin) at ($(3,0) + (ros)$) {Embedded \\system};
\node[box] (end) at ($(3,0) + (davin)$) {Endowrist};


\draw[->, ultra thick] ([yshift=0.3cm]Opt.east) -- ([yshift=0.3cm]Geo.west);
\draw[->, ultra thick] ([yshift=0.3cm]Geo.east) -- ([yshift=0.3cm]ros.west);
\draw[->, ultra thick] ([yshift=0.3cm]ros.east) -- ([yshift=0.3cm]davin.west);
\draw[->, ultra thick] ([yshift=0.3cm]davin.east) -- ([yshift=0.3cm]end.west);


\draw[<-, ultra thick] ([yshift=-0.3cm]Opt.east) -- ([yshift=-0.3cm]Geo.west);
\draw[<-, ultra thick] ([yshift=-0.3cm]Geo.east) -- ([yshift=-0.3cm]ros.west);
\draw[<-, ultra thick] ([yshift=-0.3cm]ros.east) -- ([yshift=-0.3cm]davin.west);
%\draw[<-, ultra thick] ([yshift=-0.3cm]davin.east) -- ([yshift=-0.3cm]end.west);

\node at (1.5,1) {Position};
\node at (4.5,1) {Position};
% \node at (7.5,1) {yes};
% \node at (10.5,1) {yes};

\node at (1.5,-1) {Force};
\node at (4.5,-1) {Force};
\node at (10.5,1) {Torque};
% \node at (10.5,1) {yes};
\node at (7.5,1.5) {Motor enable};
\node at (7.5,1) {Position};
\node at (7.5,-1) {Position};
\node at (7.5,-1.5) {Speed};
\node at (7.5,-2) {Current};

\end{tikzpicture}
\caption{Overall system with feedback in both direction}
\end{figure}
\todo{Do we ave a model of the operator? What is the communcation between boxes and where the f.. is the sbRIO?!?}

\subsection*{Problem formulation}
\begin{itemize}
\item \textit{How can the communication speed be increased to at least 600 Hz, such that force feedback can implement on the system?}
\item \textit{How is a controller to be designed such that it can handle delay in the system?}
\end{itemize}
\todo{Problemformulation}