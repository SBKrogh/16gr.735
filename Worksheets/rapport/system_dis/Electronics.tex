\section{Electronics}\label{sec:electronics}
% Small introduction to tikz figures and its basics, can be found on 
% http://cremeronline.com/LaTeX/minimaltikz.pdf
% Under the file tikz_magic.tex all the different boxes can be found!
%\subsection{Single board reconfigurable I/O}

\todo{Sloth reference}
The electronic setup contains a \textbf{\textit{Single Board Reconfigurable Input/Output 9636}} (sbRIO) controller which is responsible for interfacing between the surgical robot and the computer running ROS\todo{maby ref to ROS section}. It reads the sensor measurements, sends them to the PC and controls the motors based on the received reference signals. The sbRIO has built in safety protocols that disable the motors in case of error.

The controller consists of
\begin{itemize}
	\item 400 MHz real time processor
	\item 256 MB of system memory and 512 MB nonvolatile memory
	\item Reconfigurable Xilinx Spartan-6 LX45 FPGA
	\item 16 bit analog and digital I/O
	\item Built in USB, CAN, 10/100 Mb/s Ethernet peripherials
\end{itemize}

The board is configured through ethernet cable. The programs can be written in Labview, C or C++. We used LabView code to operate the sbRIO. The board is capable of running Real-time. %code, which is unmatched on the ROS side.

% http://sine.ni.com/nips/cds/view/p/lang/en/nid/210421

\subsection{Hardware setup}

The Endowrists, see \figref{fig:endowrits_set}, is actuated by 4 Maxon DC motors, see \secref{Maxon_Motor}. Each motor is equipped with an ESCON motor controller responsible for the cascade speed and inner loop current control. The control reference is sent through the sbRIO. They also provide current measurements to the sbRIO. Each motor hosts an encoder and a potmeter that provide absolute and relative angular position information to the FPGA built into the sbRIO board. For an graphical illustration of the connection see \figref{electro_setup}.
 
\todo{Complete the diagram}
\todo{som explanation for the next line UDP, or erase it?}
The sbRIO communicates with the PC using UDP protocol. 
\input{rapport/system_dis/Eletronic_component.tex}

\todo{elaborate this picture, what do the different boxes do?}


\subsection{Motor}\label{Maxon_Motor}
The four motors for actuating the Endowrist is sold as a bulk solution which include motor\cite{motor_motor}, gear\cite{motor_gear} and a position encoder\cite{motor_encoder}, see \figref{fig:Full_motor _dis}.

The control of these motors are done through the ESCON motor drivers, which are connected to the sbRIO board.\todo{ESCON motor driver needed!}

%The ESCON motor controllers are attached to the sbRIO controller and send the control signals through cables. The motor has a nominal torque of 6.96 mNm.
%4 Maxon motors are used for the actuation of the endowrist. The motors are implemented using a combination gear (Maxon 353816) including the gearing, the servo motor and the sensors. The ESCON motor controllers are attached to the sbRIO controller and send the control signals through cables. The motor has a nominal torque of 6.96 mNm.

\begin{figure}[H]
	\centering
	\begin{subfigure}{.32\textwidth}
		\vspace{0pt}
		\centering
		\includegraphics[width=\linewidth]{motor.jpg}
		\caption{The Maxon 110160 \newline DC motor}
		\label{fig:motor}
	\end{subfigure}
	\begin{subfigure}{.32\textwidth}
		\centering
		\includegraphics[width=\linewidth]{motor_gear.jpg}
		\caption{The planetary gearhead equipped with sleeve bearing}
		\label{fig:motor_gear}
	\end{subfigure}
	\begin{subfigure}{.32\textwidth}
		\centering
		\includegraphics[width=\linewidth]{motor_sensor.jpg}
		\caption{The encoder used for getting angular data}
		\label{fig:motor_sensor}
	\end{subfigure}
	\caption{The combination gear disassembled}
	\label{fig:Full_motor _dis}
\end{figure}

