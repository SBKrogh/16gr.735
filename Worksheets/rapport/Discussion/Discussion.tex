\chapter{Results and Discussion}\label{cha:discussion}

% \todo{we managed to get a closed loop, to increase speed and to estimate the force. 

% But measurements were shitty because no equipments.
% Model has oscillations because it feedback small forces.
% Friction model improves that from the simulation we made and should be even better when used.
% +Philipe 

% Communication does not reach the aimed refresh rate so force feedback doesn't either.
% Compression should help if it is fast.
% Code should be optimizing by using lower level libraries for communication on linux.
% Safety should be considered more than what we did.
% Consider implementing a real time system like RTAI.

% }

\section{Results}


\todo{The results about communication were already stated in Communication, should they be moved?}

An operator control uses the Geomagic Touch to clamp a finger. As no equipment is available for measuring the force applied by the end-effector, the estimate force is compared to the error in position and the current applied to the actuator. The experiment is made for the maximum frequency reached of 638 Hz in \figref{fig:fbkm}, the operator clamps the finger slowly. For comparison, measurements for the original frequency of 100 Hz are displayed in \figref{fig:fbkm_100}, the operator clamps and releases the finger at a fast pace.


\begin{figure}[H]
  \input{../Worksheets/rapport/pictures/feedback_measurements_copy_worksheets}
  \caption{Clamping of a finger with a refresh rate of 638 Hz for the communication}
  \label{fig:fbkm}
\end{figure}

\begin{figure}[H]
  \input{../Worksheets/rapport/pictures/clamp_feedback_more_mvt}
  \caption{Clamping of a finger with a refresh rate of 100 Hz for the communication and a varying command}
  \label{fig:fbkm_100}
\end{figure}

Similarly to the experiment described for the gripping force, the operator uses the Geomagic Touch to control the roll and the mouvement of the hand effector are hindered by synthetic skin, the resulting measurements are represented in \figref{fig:fbkm_roll_100}.
\begin{figure}[H]
  \input{../Worksheets/rapport/pictures/roll_feedback_measurement}
  \caption{Response of the force feedback for the roll for varying command with a refresh rate of 100 Hz for the communication}
  \label{fig:fbkm_roll_100}
\end{figure}


\todo{add pitch is unstable / shitty, i don't know why}


\section{Discussion}


From \figref{fig:fbkm}, the yaw force fed back to the user by the Geomagic Touch dynamically corresponds to both the current increase and the position error. From \secref{sec:friction}, the friction value computed for the gripping force are 0.20 and 0.21 A. This value is overcome at 2 s and the force fed back starts to increase at 2.2 s. Thus a delay of 0.2 s between applying force to the external environment and receiving feedback is present. Considering that the communication loops have a refresh rate of 638 Hz and 1000 Hz, amounting to a delay of 2.6 ms, the delay resulting from the communication can be neglected in front of the one resulting from the model. In addition, delay of 0.7 s is present when releasing the finger.
From \figref{fig:fbkm_100}, it can be seen that with a refresh rate of a 100 Hz the model can still estimate the dynamics of the force applied to the end-effector even when varying the command sent.

The roll model of a lower order can accurately track the dynamics of the force without showing the same delay, however it is more sensitive to noise as it can be seen at 6 s in \figref{fig:fbkm_roll_100}.

\todo{add stuff about pitch}

In \figref{fig:fbkm}, the operator attempted to reach a point with no feedback outside of the main clamping. However it can be seen than before 2 s and after 11 s, the operator moved trying to find that point and received feedback at that time. The first deduction is that the mapping and scaling of the command should not allow to attempt to open the clamp more that it is possible as the current increases when trying to exceed the range of movement possible. This conclusion was supported by the negative feedback visible in \figref{fig:fbkm_100}. The second deduction was that an accurate model of the nonlinearities was required. A model of friction was derived in \secref{sec:friction} and promising results were obtained in simulation. It is believed that implementing this model would reduced the oscillations even further than it does in the simulation as the operator would have less difficulty in finding the point were there is no feedback.

Although the refresh rate of the communication does not allow to reach the goal of 550 Hz for the feedback loop, a significant improvement can be noted.  To further increase the refresh rate three axis are considered: compressing data, optimizing the programs and implementing on a real-time system.\\
Compression of exchanged data was evoked in \secref{subsec:minimizing}. It is believed that the implementation of a fast compression algorithm would improve the refresh rate in a communication between two computers. However additional precautions must be taken when implementing one the embedded system as the computation power is not as high as it is for a computer. The computation time required to compress the small amount of data sent should be compared to the transmission time of said data.\\
From \cite{million_packets} it is possible to receive a million packets per second on a Linux system. Although such a high frequency cannot be reached when other tasks have to be performed on the computer, it should be possible to increase the frequency by optimizing the programs used for the project. Using low level libraries for communication would be the first step to improve the communication. In addition, the program currently runs many thread to communicate with external devices and between internal processes. These threads are not synchronized or prioritized, thus a new software architecture could reduce the load on the processing unit and ensure all threads can be properly executed.\\
The last axis considered is an extension of the previous suggestion to create a new software architecture to increase control over the different thread. It is believed that implementing the force feedback on a real-time system such as Real Time Application Interface (RTAI). By using a real-time system, better control over the priority of each process can be achieved and thus less computation power can be allowed to processes non-essential for the force feedback such as the graphical interface.\\
Another aspect that should be considered for future works is the safety in the communication. Currently only a simple detection of connexion timeout has been implemented. In future application, security against cyber-attacks should be considered as the system could be extended to remote teleoperation.
