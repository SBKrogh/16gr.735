This project covers the development of a controller system for a model container crane. The goal of this project is to automatically relocate containers fast, reliably and efficiently. To achieve this, the hardware platform has been improved and is now based on digital motor drivers together with a FPGA as the main controlling unit. The control system is based on multiple controllers derived by the use of classical control methods, namely the root locus method. The system models describing the behavior of the crane are derived using a combination of Newtons law's and free body diagrams to describe this graphically. An issue discovered during the project, is that the static friction is not constant along the axes of the setup. In order to compensate for the static friction a current offset is implemented. However, this yields issues as the static friction is not constant. The controller system for relocating containers has been proved functional.Additionally an interface has been implemented, this allows to relocate containers from eight different positions along the x-axis and three locations along the y-axis.          