\section{ROS}\label{sec:ros}

In order to communicate between ROS and the sbRIO board the university created a ROS node called davinci\_driver. This driver is composed of three parts:
\todo{The university didn't create anything!}

\begin{itemize}
\item Low level driver
\item Middle level driver
\item High level driver
\end{itemize}
\todo{Itemizer need reference to Git repo}

The low level driver communicates directly with the sbRIO board. The high level driver handles the communication between ROS nodes. It updates the data for the nodes and transmits the setpoints to the low level driver through the middle driver. The name "setpoint" is used to designate the next position a motor should take.

The middle level driver handles both the creation of the low lever driver(s) and the communication between the low and high level driver so that there is no need for the client, ROS in our system, to acquire mutexes. Since it is possible to create more than one low level driver it is possible to communicate with more than one arm of the DaVinci robot.

The low level driver connects to one sbRIO board using a TCP/IP socket and exchange data with it using \gls{JSON} files, see \secref{subsec:JSON}. As shown in ~\figref{fig:original_driver}, the driver begins with an initialization of the connection and then start a loop. This loop is started in a thread and will run as long as the node is running. At each iteration of the loop, if some new data are present in the stream they are read and an update function is called to make them available to the higher levels. Then if there is some new setpoints or motor enable they are sent to the sbRIO. Finally, the thread goes to sleep before the next iteration of the loop. This sleeping period is introduced as to not overload the communication channel or the processor and this is the only control available on the speed of the communication.

\begin{figure}[H]
\centering
\begin{tikzpicture}

\node[box] (Initialization) at (0,0) {Initialization};
\node[box] (Receive) at ($(0,-2)+(Initialization)$) {Read the data \\if some were received};
\node[box] (Send) at ($(0,-2)+(Receive)$) {Send if the\\ control changed};
\node[box] (Sleep) at ($(0,-2)+(Send)$) {Sleep};
\node[box] (Update_State) at ($(5,0)+(Receive)$) {Update the data\\ available for higher\\level processes};

\draw[->, ultra thick] (Initialization) -- (Receive);
\draw[->, ultra thick] (Receive) -- (Update_State);
\draw[->, ultra thick] (Receive) -- (Send);
\draw[->, ultra thick] (Send) -- (Sleep);
\draw[->, ultra thick] (Sleep.west) -| ++(-2,0) |- (Receive);

\end{tikzpicture}
\caption{Structure of the original sbRIO driver}
\label{fig:original_driver}
\end{figure}


% The middle level driver allows the creation of more than one low level driver. Due to this it is possible to communicate with more than one arm of the DaVinci Robot at once. It also handles the communication between the high level driver and the low level so that there is no need for the client to acquire mutexes.\todo{last sentence - half of it is double confetti}