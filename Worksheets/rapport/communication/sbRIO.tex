\subsection{sbRIO}\label{sec:sbrio}

\todo{it would be nice to have a simple block diagram of the TCD, Data out/in and Debug/data loop. If we have the time do this!}

The sbRIO code running in the former setup has four code loops running in parallel. Since the sbRIO uses a single-core processor, the execution order depends on the operating system's decision, and can have stochastic characteristics. The four separated loops have distinct functions:

\begin{itemize}
	\item TCP connection loop for communicating with the ROS node
	\item Data out loop for reading data out of the FPGA
	\item Data in loop for setting FPGA parameters according to the received JSON packet
	\item Debug/data logging loop
\end{itemize}

The debug/data logging loop is not used during normal execution. The TCP loop communicates with the data loops through queues. It uses one queue for the received data, one for data out. The queues have only one element. When an enque attempt is made, the previous element is discarded and the newer one takes its place, ensuring that the queue always has the most up-to-date element.

The TCP communication is set up serially, i.e. the sbRIO sends data then waits for incoming data. It is a timed loop, thus the code attempts to run once in each specified period. When the loop is executing for the first time or when it experiences error, it creates/recreates a TCP connection using a TCP listener. After that, the loop reads the data out queue, if it finds an element in it, it sends it through the TCP connection. Then it polls for incoming data. When it arrives, the loop enques the data in the received data queue. Then it restarts the process.

When the user stops the execution, the connection is terminated.