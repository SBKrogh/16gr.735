\section*{Minimum refresh speed}\label{sec:min_speed}
\todor{Somebody has to accept this chapter}
\todom{Some place else}
The refresh rate for the haptic feedback has an influence on how transparent the control will be for the user. An ideal situation is that there is no delay in the system and the system is correctly modeled, thus give the user a direct feedback of what force the tool are exposed to. 

As one of the goals for this project is to give a higher bandwidth for the communication and thereby enable the opportunity of making a more transparent feedback control, the minimum required refresh rate should be found.

From the article \textit{The role of haptics in medical training simulators: a survey of the state of the art}\cite{coles2011role}, it is said that the minimum refresh rate of haptic feedback is wildly debated to be between 300 Hz and 600 Hz, but for a realistic force feedback it is commonly accepted to be at least 1000 Hz, thus it is not defined. 

From \chapref{cha:prev_communication}  the communication between the sbRIO and the computer is in the present setup at a 100 Hz, which does not meet the minimum of 300 Hz. Therefore the speed should be raised with at least a factor of three. It is however preferable to exceed this so the system does not run on the limit.
\todoc{Should we specify our goal more precisely than this? (>300 Hz is a loose definition)}
