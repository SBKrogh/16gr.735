% When increasing the communication speed, increasing jitter and packet loss occur as these parameters are highly correlated to
% the network congestion. %\cite{cisco_jitter}
% The maximum refresh rate reached does not meet the generally accepted speed requirement of haptic devices \cite{coles2011role}, but the discontinuities occuring are practically impossible to notice.
% To further increase the refresh rate, compression of the data
% was considered and it is believed that implementing a fast
% compression algorithm such as the one described in \cite{fast_ZIV} could reduce the time required to transmit a packet.
% Due to the structure of the EndoWrist, we have split the
% dynamical model of the force into separate submodels pertaining
% to the actions most commonly performed.

\subsection*{Communication}

When increasing the communication speed, increases in jitter and packet loss occur as these parameters are highly correlated to the network congestion. %\cite{cisco_jitter}
The decreasing jitter and sharp increase in packet loss at the highest frequency can be explained by the trade-off between frequency and packet discarding that was made when implementing.
The maximum refresh rate reached does not meet the generally accepted speed requirement of haptic devices \cite{coles2011role}, however the maximum refresh rate was increased from 100 Hz to 638 Hz.
To further increase the refresh rate, compression of the data was considered and it is believed that implementing a fast compression algorithm such as the one described in \cite{fast_ZIV} could reduce the time required to transmit a packet.

\subsection*{Force estimation}

The estimated force could not be compared to the real force applied by the end effector as no suitable force sensor was available for this project. Instead, the estimated force is compared to current applied to the actuator and error in position. From Fig. \ref{}, it can be seen that the estimated force follows the trend of both the current and the position error when clamping. However some oscillations in the feedback are present even when the user is not clamping anymore. Promising results have been obtained in simulation by using a friction model to determine when there are external forces. When the friction is overcome, the estimated force is fed back, otherwise no feedback is sent.