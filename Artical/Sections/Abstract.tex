\begin{abstract}

Haptic feedback is a form of transferring information to the user via the sense of touch, usually through some sort of input device the user gives commands with.
This makes it ideal for teleoperating tasks requiring precision in applied force, robotic surgery being a prime example.
Currently, haptic feedback in teleoperation is subject to numerous constraints on time delay and accuracy.
Nonetheless, results show that implementing this type of feedback in teleoperated robotic surgery gives vastly better results.
In this paper we propose a new method of teleoperating the DaVinci surgical robot using haptic feedback.
The method involves using a state-of-the-art haptic device to control a surgical tool serving as the robots end-effector.
Since the dynamics of the surgical tool are strongly non-linear, estimation techniques are used to calculate reaction forces on the device.
Changes are made to existing communication protocols in order to reduce time delay\todo{Maybe add results (operating frequency for example)}, which is an important factor.
The various constraints and challenges are adressed individually in each section. \todo{Erase last sentence}
\end{abstract}



