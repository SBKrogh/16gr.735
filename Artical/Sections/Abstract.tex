\begin{abstract}

Haptic feedback is a way of transferring information to the user via the sense of touch, usually through the same input device the user gives commands with.
This makes it ideal for teleoperating tasks requiring precision in applied force, robotic minimally invasive surgery (MIS) being a prime example.
Currently, haptic feedback in teleoperation is subject to numerous constraints on time delay and accuracy.
Nonetheless, results show that implementing this type of feedback in teleoperated robotic surgery results in a higher successes rate compared to the traditional robotic MIS.
%gives vastly better results.
In this paper, we focus on improving the haptic feedback on the da Vinci robot at Aalborg University using the existing hardware.
The method involves using a state-of-the-art haptic device to control a surgical tool serving as the robot’s end-effector.
Since the dynamics of the surgical tool are strongly nonlinear, estimation techniques are used to calculate reaction forces on the device.
Changes are made to the existing communication protocols in order to reduce time delay.

%The various constraints and challenges are addressed individually in each section. \todo{Erase last sentence}
\end{abstract}



