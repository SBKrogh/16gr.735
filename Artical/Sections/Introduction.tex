\section{Introduction}\label{sec:introduction}
% no \IEEEPARstart
\todo{include other methodes (vibro-tactil) + others force feedback works}
Due to the advantages\todo{still disputed} of robotic surgery, the interest in surgical robots has grown\cite{forbes}.
Increased precision provided by surgical robots introduces a decrease in tissue damage, thus reducing the recovery time\cite{RIGSP}.
% Over the last couple of years the interest of surgical robots has grown\cite{forbes}.
% %Over the last couple of decades the use of robots for minimally invasive surgeries has become more widespread. %interest growth
% Some advantages of robotic surgery are the precision and reduction e.g in tissue damage, thus reducing the recovery time\cite{RIGSP}.
Robots used in these surgical procedures have an attached end-effector that is used as a surgical tool.
One such tool is the EndoWrist.
The main advantage of the EndoWrist lies in its construction, as it is made to be manipulated in a similar manner to the operators wrist.
%Commonly the feedback the operator are the visual cues in the live video transmission of the surgery.
%Commonly the feedback received by the operator consists of visual cues in the video transmission of the surgery.
During surgery the operator receives \textcolor{green}{3D visual} feedback from the control loop. \textcolor{red}{Commonly this feedback consists of visual cues in the video transmission of the surgery.}


The problem with the operator having exclusively visual feedback lies in the fact that the surgeon has to estimate the force applied by observing the \textcolor{green}{color changes in the video feed caused by the } deformation \todo{it changes color} of the skin and organs for each maneuver.
 %\textcolor{green}{



This constant effort from the operator not only increases the operation time but also leads to errors such as thread breaking during stitching or damaging tissues by applying too much force,\cite{lee2015grip}. It has been shown experimentally that haptic feedback has a considerably positive effect on the reduction of surgical error\cite{EOFGFF}.


The purpose of haptic feedback in Robotic MIS (RMIS) is to restore the sense of touch of the surgeon. One way of doing so is by applying forces on the controller of the surgeon that replicate the forces applied on the end-effector. This method is named force feedback and in an ideal case would give the same feeling as if directly holding a surgical tool. The direct force feedback calculates the feedback from the resistance affecting the actuators. \textcolor{red}{However, as the tool is highly nonlinear, a significant part of the force applied to the controller of the operator corresponds to the nonlinearities.}
\textcolor{green}{However, as the tool is highly nonlinear, the output power is lower than the input power, the rest is absorbed by the tool's inherent damping.}
 Any forces related to the construction of the robot are not desired in the feedback as the operator would not feel them when holding a tool. In order for the operator to feel as if he was directly holding the tool the control system should be transparent to him.

%{\color{red}
%The purpose of force feedback in Robotic MIS (RMIS) is to apply forces that resist the operator’s movements to match the external forces applied to the end-effector.
%Force feedback gives the mechanosensory sensation of directly manipulating the end-effector, thus reducing the number of surgical errors.


The haptic feedback could be done as direct force feedback calculated from the resistance affecting the actuators, but as the tool is highly nonlinear, the transparency of the controller would suffer from it. It would be possible to solve this problem by implementing a sensor on the end-effector to measure the force, but due to the demand for high hygiene, the tools have to be sterilized at temperatures over a $100^\circ$ C which could damage the sensor. Furthermore it is stated by law\todo{what law?} that each surgical tool has to be discarded after a few use. This means that the cost of the tool has to stay as low as possible and therefore make the idea of implementing an expensive sensor not ideal. Therefore\todo{cannot start paragraph with Therefore - i just merged the two paragraphy} the force feedback has to be estimated through the actuators, which requires a dynamic model of the tool. From this model the forces related to the actuation of the tool can be estimated and the external forces applied at the end-effector can be computed. Although, accuracy of the force feedback is extremely important\todo{is it?} for transparency, the response time and the frequency of the control loop also have to be considered as any of those could break transparency if too high. The frequency of the control loop is directly related to the frequency of the communication between the different components of the system. It is widely discussed what the minimum refresh rate of the feedback loop should be but seems to be someplace between 300 Hz and 1000 Hz depending of the hardness of the object\cite{coles2011role}.%}

%\textcolor{red}{Therefore the force feedback has to be estimated through the actuators, which gives a high performance demand for the feedback controller as it has to be as precise as possible to feedback the correct force to the operator. Another important subject is the transparency of the feedback, as the operator should have the feeling of doing the operation by hand and not remotely. This puts a demand on the speed of the feedback loop, as the faster the loop runs the smoother the force feedback to the operator will feel. This will set demands on communication frequency since we need to keep time delay to a minimum. It is widely discussed what the minimum refresh rate of the feedback loop should be but seems to be someplace between 300 Hz and 1000 Hz depending of the hardness of the object\cite{coles2011role}.}

%\textcolor{green}{
In this paper, an attempt is made to implement a force feedback on a setup emulating the essential parts of the da Vinci Robot without implementing new hardware. To do so, a dynamic model is derived and a control strategy is proposed. The aimed frequency for the feedback loop is 600 Hz, to reach that goal analysis of the communication protocol is provided. In section II, we will take an overview of our proposed control system as a whole, briefly presenting each of the components and their interaction.
Section III will cover the methods used to create a dynamic model of the EndoWrist and proposed methods of translating the estimated force to actual force fed back to the operator.
Section IV contains descriptions of various problems pertaining the requirement of transparency and explanations of methods used to address them.
Finally,  we present (the expected) experimental results in Section V and and draw a short conclusion in Section VI.




% \section{Introduction}
% % no \IEEEPARstart
% Over the last couple of decades the use of robots for minimally invasive surgeries has increased.
% This is mainly due to its precision and the reduction in tissue damage\cite{RIGSP}.
% Robots used in these surgical procedures have an attached end-effector that is used as a surgical tool.
% One such tool is the Endowrist. 
% The main advantage of the Endowrist lies in its construction, as it is made to be manipulated in a similar manner to a human wrist.

% Currently the only feedback operators get is visual, using live video transmissions of the surgery. 
% The problem with the operator having exclusively visual feedback lies in the fact that the surgeon has to estimate the force applied by the deformation of the skin and organs for each maneuver. %to skin and organs in every single manoeuvre made. 
% It has been shown experimentally that haptic feedback has a considerably positive effect on the reduction in surgical error\cite{EOFGFF}. 

% The purpose of haptic feedback is to apply force that resist the operators movement in such a way that it represents some information about the environment or situation.
% In theory, this should make teleoperated surgical procedures a lot more realistic to the operator and thus reduce accidents.

% The haptic feedback could be done as direct force feedback calculated from the resistance on the actuators but as the tool is highly non-linear the transparency of the controller will become less transparent\todo{transparent x2}.
% It would be possible to solve this problem by implementing a sensor to the end effector to measure the force but due to the demand of high hygiene the tools have to be sterilized at temperatures over a $100^\circ$ C which could damage the sensor(s). Furthermore it is stated by law that each surgical tool has to be discarded after a few times in use\todo{after a few use}. This means that the cost of the tool has to stay as low as possible and therefore make the idea of implementing an expensive sensor not ideal.

% Therefore the force feedback has to be estimated through the actuators which gives a high demand for the feedback controller as it has to be as precise as possible to feedback the correct force to the operator. Another important subject is the transparency of the feedback as the operator should have the feeling of doing the operation by hand and not remote\todo{remotely or through a robot}. This puts a demand on the speed of the feedback loop as the faster the loop runs the more smooth the force feedback to the operator will feel. This will set demands on communication frequency since we need to keep time delay to a minimum. It is wildly discussed what the minimum refresh rate of the feedback loop should be but seems to be someplace between 300 Hz and a\todo{remove "a"} 1000 Hz\cite{coles2011role}. 

% In the section II we will take an overview of our proposed control system as a whole, briefly presenting each of the components and their interaction.
% Section III will cover methods used to create a dynamic model of the Endowrist and proposed methods of translating the estimated force to actual force fed back to the operator.
% Section IV contains descriptions of various problems pertaining the requirement of transparency and explanations of methods used to address them.
% Finally,  we present (the expected) experimental results in section V and and draw a short conclusion in section VI.
