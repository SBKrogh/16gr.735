\section{Improvements}
%%%%%%%%%%%% MID WAY AGENDA %%%%%%%%%%%%%%
\begin{frame}<beamer>
\frametitle{State estimation}
  \begin{itemize}
    \item<1-> Modeling for additional outputs allows correction of the model using an estimator
    \item<2->  A multiple output model that adequatley captures the dynamics of the system could be used in a Kalman filter to create a state estimate
    \item<3-> The state estimates can be used in a state feedback loop to change system dynamics
    \item<4-> This means that reference following capabilities can be added to the system, dispite the nonlinear characteristics of the dynamics
  \end{itemize}
\end{frame}

\begin{frame}<beamer>
\frametitle{State estimation}
  \begin{itemize}
    \item<1-> The hypothesis was tested in simulation
    \item<2-> Simulation results show that full reference following is possible despite the input nonlinearities in the system
    \item<3-> While the transient behaviour of the reference value is replicated, offsets and parasitic gains need to be compensated
    \item<4-> Could be implemented with improved model, doesn't improve estimate of current one.
    \end{itemize}
\end{frame}